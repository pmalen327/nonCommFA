\section{Preliminaries}
\label{sec:prelim}
\subsection{Definition of an Algebra}
\begin{definition}
    An \textbf{algebra} is a vector space $A$ with a multiplication operator
    $A \times A \to A$ and $(x,y) \mapsto xy$ such that
    \begin{enumerate}
        \item $(xy)z = z(yz)$ for all $x,y,z \in A$
        \item $x(y+z) = xy + xz$ and $(x+y)z = xz+yz$ for all $x,y,z \in A$
        \item $c(xy) = (cx)y = x(cy)$ for all $c \in \mathbb{C}$ and $x,y \in A$
    \end{enumerate}
\end{definition}
We say $A$ is \textbf{unital} if it has an identity element $e \in A$ such that
$ex = xe = x$ for all $x \in A$. We will assume every algebra is unital unless stated
otherwise. Further, $A$ is \textbf{commutative} or (\textbf{abelian})
if $xy=yx$ for all $x,y \in A$.

\subsection{Definition of a Banach Algebra}
\begin{definition}
    A \textbf{Banach algebra} $A$ is simultaneously an algebra and a Banach space satisfying
    \begin{equation*}
        \| xy \| \le \|x\| \|y\| \quad \forall x,y \in A
    \end{equation*}
\end{definition}
If $A$ is unital then we assume $\| e \| = 1$. Further, we usualy write $1:=e$ for
the unital element. We won't work much with Banach algebras exclusively but rather
to motivate the definition of a C*-algebra. However, it is important to note an example.
\begin{example}
    Let $f,g \in L^1(\mathbb{R})$ and define the algebra's operation as the convolution
    $f * g \in L^1 (\mathbb{R})$ by
    \begin{equation}
        (f * g)(t) := \int_{\mathbb{R}} f(t-s)g(s)ds
    \end{equation}
    Under the convolution, $L^1 (\mathbb{R})$ is a commutative Banach algebra but
    it is not unital.
\end{example}



\subsection{Definition of a C*-Algebra}
\begin{definition}
A \textbf{C*-algebra} is a Banach algebra \( A \) equipped with an involution \( a \mapsto a^* \) satisfying the C*-identity:
\[
\|a^*a\| = \|a\|^2 \quad \text{for all } a \in A.
\]
\end{definition}
Note that the involution operator is unique. Most C*-algebras are endowed with the
opertor norm-induced topology and we will always assume this to be the case unless
stated otherwise. For some concrete examples and nonexamples,
consider a compact Hausdorff space $X$. We define $C(X)$ to be all continuous functions
$f: X \to \mathbb{C}$. It is crucial that $X$ be compact. The involution on $C(X)$
is defined as complex conjugation denoted $f^* = \overline{f}$. Lets look at some
simple examples.
\begin{example}
    Let $X = [0,1]$, then $C(X)$ is a C*-algebra.
\end{example}

\begin{example}
    Let $X = S^1$, the unit circle. Then $C(X)$ is a C*-algebra.
\end{example}

\begin{example}
    Let $X = \{ 0, 1, 2, 3\}$. Then $C(X)$ is a C*-algebra.
\end{example}

\begin{example}
    Let $X = \mathbb{R}$. Then $C(X)$ is not a C*-algebra because $X$ is not compact.
\end{example}

These are very basic concrete examples of subsets of $\mathbb{C}$, but we also want
to consider some much more important examples for our purposes. Arguably the most important
example is the C*-algebra $C_0(X)$, the C*-algebra of continuous functions that vanish
at infinity over a locally compact Hausdorff space $X$. This leads us to one of the most
important and useful theorems in the theory of operator and C*-algebras. First, a quick
definition.
\begin{definition}
    For a C*-algebra $A$, a \textbf{character} is a nonzero homomorphism
    \begin{equation*}
        \Phi:A\to \mathbb{C}
    \end{equation*}
\end{definition}

\begin{theorem}[Gelfand Representation Theorem]
    Every commutative unital C*-algebra  $A$  is isometrically isomorphic to  $C_0(X)$,
    the algebra of continuous functions on some compact Hausdorff space $X$, where
    $X$ is the space of characters.
\end{theorem}
\vspace{3mm}
The utility of the Gelfand Representation Theorem is quite apparent. This allows
us to view all commutative C*-algebras as isomorphic. Specifically, for any commutative
unital C*-algebra $A$, there exists an isometric isomorphism
\begin{equation*}
    \varphi: A \to C_0(X)
\end{equation*}
The importance of this theorem cannot be overstated. This property of unital commutative
C*-algebras is often referred to as \textbf{Gelfand duality}.
A valuable C*-algebra that shows up a lot in physics is the C*-algebra of complex-valued $n \times n$
matrices, denoted $M_n(\mathbb{C})$. Another C*-algebra of particular interest is
the bounded linear operators from a Hilbert space onto itself, we denote this $B(H)$
for a Hilbert space $H$.

\par

In summary, and with some abuse of notation, we can see how much more structure
C*-algebras have as compared to typical vector spaces.
\begin{equation*}
    \text{vector space} \supseteq \text{algebra} \supseteq \text{Banach algebra} \supseteq \text{C*-algebra}
\end{equation*}

\subsection{States on C*-Algebras}
\begin{definition}
A \textbf{state} on a C*-algebra \( A \) is a linear functional \( \phi: A \to \mathbb{C} \) such that:
\[
\phi(a^*a) \geq 0 \ \  \text{for all } a \in A, \ \  \text{and } \phi(1) = 1.
\]
\end{definition}


We require the positive semidefinite condition for $\phi$ to preserve the positive structure
of $A$. We define the set of all the states on $A$ as the \textbf{state space} $S(A)$. It
can be shown that $S(A)$ is compact, non-empty, and convex (state spaces are ``nice'').
There are certain states that are of particular interest. 
We say a state $\phi \in S(A)$ is \textbf{pure} if it is an extreme point of $S(A)$.
Note this is well-defined because the Krein-Milman theorem guarantess the existence
of extreme points of $S(A)$. States that are not pure are \textbf{mixed}.

\par

States of course have the typical properties that we'd expect from a linear functional
but they also admit some other properties that may not be so apparent. 

\begin{definition}
    The \textbf{kernel} of a state $\phi$ is defined as
    \begin{equation*}
        \text{ker}(\phi) = \{a : \phi(a^*a) = 0 \}
    \end{equation*}
    for $a \in A$, a C*-algebra. Furthermore, a state $\phi$ on a C*-algebra
    is \textbf{faithful} if $\phi(a^*a) = 0$ implies $a=0$ for all $a \in A$.
\end{definition}
It is very important to note that this is different than the kernel of a general
linear functional. In a sense, this tells us that faithful states ``detect'' non-zero
elements. As we will see later, a state is faithful if and only if its GNS representation
is injective. States have many other properties of interest, particularly
\begin{itemize}
    \item The state space $S(A)$ is compact in the weak-* topology
    \item A state is \textbf{tracial} if $\phi(ab) = \phi(ba)$ for all $a,b \in A$
    \item Pure states correspond to rank-1 projections onto a unit vector $\psi \in H$.
    Specifically, $\phi(A) = \langle \psi, A\psi \rangle$
    \item A mixed state is a convex combination of pure states
\end{itemize}