\section{Applications and Connections}
\subsection{Quantum Mechanics}
States are a key concept in quantum mechanics. We saw earlier how states on $B(H)$
assign probability distributions to observables. We also looked at the trace state
on the algebra $M_2(\mathbb{C})$ and saw how it relates to density matrices. Now,
we will look at some non-trivial states in quantum mechanics that are of equal
or greater importance. We will start with some background on some important operators,
then see the C*-algebras they generate and ultimately work through the GNS construction
of the Fock spaces.
\subsubsection{Weyl and Majorna Operators}
First we introduce the bosonic creation $a^\dagger(f)$ and annihilation $a(f)$ operators.
If these operators satisfy the \textbf{canonical commutation relations}:
\begin{align}
    [ a(f), a^\dagger(g) ] &= \langle f,g \rangle\\
    [a(f), a(g)] &= 0\\
    [a^\dagger(f), a^\dagger(g)] &= 0
\end{align}
Where $[- , -]$ is the Lie bracket. Then we can define the \textbf{Weyl operator}:
\begin{equation}
    W(f) = e^{i(a(f)+a^\dagger(f))}
\end{equation}
Where $f$ and $g$ are functions from a hilbert space $H$. Weyl operators satisfy
the \textbf{Weyl commutation relations} given by:
\begin{equation}
    W(f) W(g) = e^{-\frac{i}{2} \operatorname{Im} \langle f, g \rangle} W(f+g)
\end{equation}
Weyl operators satisfy a lot of convenient properties but it is sufficient to note they
are unitary operators, specificially:
\begin{align*}
    W(f)^* &= W(-f)\\
    W(0) &= 1
\end{align*}
From a physics perspective, Weyl generate phase space transformations. They also
describe coherent states and quantum optical transformations. These are beyond the 
scope of this survey but it is important to note that Weyl operators play a key
role in physics.\newline

\noindent We saw that Weyl operators satisfy the canonical commutation relations and thus
the question arises; are there anticommutation relations and further, corresponding
operators? First, for the creation and annihilation operators, we let $c(f)$ and $c^\dagger(f)$
be the fermionic creation and annihilation operators respectively. We define the
\textbf{canonical anticommutation relations} under the anticommutor operator $\{A,B\} = AB + BA$
(sometimes called the Jordan bracket, the positive analogue of the Lie bracket):
\begin{align}
    \{ c(f), c^\dagger (g)\} &= \langle f,g \rangle\\
    \{ c(f), c(g)\} &=0\\
    \{ c^\dagger(f), c^\dagger(g)\} &= 0
\end{align}
Given these relations, we can define the anticommutative analogue to Weyl operators.
The \textbf{unitary fermionic coherent operator} is defined as:
\begin{equation}
    U(f) = e^{c^\dagger(f) - c(f)}
\end{equation}
However, the exponential is usually dropped and the unitary fermionic coherent operator
becomes the \textbf{Majorna operator}:
\begin{equation}
    \gamma(f) = c(f) + c^\dagger(f)
\end{equation}
Majorna operators satisfy the \textbf{Clifford algebra relations}:
\begin{equation}
    \{ \gamma(f), \gamma(g) \} = 2 \operatorname{Re} \langle f, g \rangle
\end{equation}
Majorna operators are used in fermionic systems, particularly in topological phases
of matter and quantum computing.

\subsubsection{CCR and CAR Algebras}
It might not be obvious how Weyl and Majorna operators show up in the theory of C*-algebras
but now we demonstrate how they generate two of the most important algebras in
physics. Given a Hilbert space $H$, the Weyl operators on $H$ generate the
\textbf{Canonical Commutation Relations algebra (CCR)}. Under the Weyl commutation relations,
the CCR algebra is a C*-algebra.

\subsubsection{Fock Spaces as Representations}
Fermionic and Bosonic
\subsection{Noncommutative Geometry}
Briefly describe the role of states in measuring noncommutative spaces.