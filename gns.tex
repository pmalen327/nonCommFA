\section{The Gelfand-Naimark-Segal (GNS) Construction}
\subsection{Overview of the Construction}
The Gelfand-Naimark-Segal (GNS) Construction is a very important method in the
theory of operator algebras and related fields. On a high level, the idea is to 
take a state $\phi$ on a C*-algebra and turn it into a representation of the whole
algebra over a Hilbert space. First, we define representations (technically these
are *-representations).
\begin{definition}
    A \textbf{representation} of a C*-algebra $A$ over a Hilbert space $H$ is a map
    $\pi: A \to B(H)$ such that
    \begin{itemize}
        \item $\pi$ is a ring homomorphism and carries the involution from $A$
        into the involution on operators in $B(H)$
        \item $\pi$ is nondegenerate and thus unit-preserving ($\pi(1) = 1$)
    \end{itemize}
\end{definition}

\begin{definition}
    Let $\pi$ be a representation of a C*-algebra $A$ on a Hilbert space $H$. An
    element $\xi$ is called a \textbf{cyclic vector} if the set of vectors
    \begin{equation*}
        \{ \pi(x) \cdot \xi : x \in A \}
    \end{equation*}
    is norm dense in $H$. Specifically, the range of the representation $\pi(x) \xi$
    is dense in $H$ with respect to the induced norm. When this is the case, $\pi$ is called a \textbf{cyclic representation}.
\end{definition}


The ultimate goal of the GNS construction is to take an abstract state
and realize it as a representation on a Hilbert space. Lets do this step by step.
\begin{enumerate}
    \item We take the algebra $A$ and define a pre-Hilbert space on it where the vectors
    in the pre-Hilbert space are equivalence classes of elements of $A$ with inner-product
    $\langle a, b \rangle_\phi = \phi(b^*a)$ for a state $\phi$ on $A$.
    \item Now the algebra $A$ can act on this pre-Hilbert space by operators. For
    each $a \in A$ we define the operator (representation) $\pi_\varphi (a)$ as acting
    on the Hilbert space $H$ (the completion of the pre-Hilbert space). This representation
    is explicitly realized as $\pi_\varphi(a) \cdot x = a \cdot x$ for some $x$ in
    the Hilbert space $H$.
    \item The construction of $H$ gives us a cyclic vector $\xi \in H$ generated by
    the identity of $A$. This vector is crucial because all of $H$ is generated
    by the action of $A$ onto this vector. Particularly, all vectors in $H$ can
    be written $\pi_\phi(a) \cdot \xi$ for some $a \in A$.
    \item The state $\phi$ corresponds to the expectation of the representations
    $\pi_\phi(a)$ acting on $\xi$. Specifically, $\phi(a) = \langle \pi_\phi(a) \cdot \xi, \xi \rangle$.
\end{enumerate}
Thus our state $\phi$, which started life as a linear functional, can now be realized
as a real-valued expectation on $H$. Now we have all the ingredients we need to
explicitly define the GNS construction.

\begin{theorem}[Gelfand-Naimark-Segal (GNS) Construction]
    Given a state $\phi$ of $A$, there is a *-representation $\pi$ of $A$ acting
    on a Hilbert space $H$ with a cyclic vector $\xi$ such that
    \begin{equation*}
        \phi(a) = \langle \pi(a)\xi , \xi \rangle
    \end{equation*}
    for every $a \in A$.
\end{theorem}


\subsection{Examples}
\begin{example}
GNS construction for \( C([0,1]) \) with a specific state.
\end{example}
\begin{example}
GNS construction for \( M_2(\mathbb{C}) \).
\end{example}