\section{The Gelfand-Naimark-Segal (GNS) Construction}
\subsection{Overview of the Construction}
The Gelfand-Naimark-Segal (GNS) Construction is a very important method in the
theory of operator algebras and related fields. On a high level, the idea is to 
take a state $\phi$ on a C*-algebra and turn it into a representation of the whole
algebra over a Hilbert space. First, we define representations (technically these
are *-representations).
\begin{definition}
    A \textbf{representation} of a C*-algebra $A$ over a Hilbert space $H$ is a map
    $\pi: A \to B(H)$ such that
    \begin{itemize}
        \item $\pi$ is a ring homomorphism and carries the involution from $A$
        into the involution on operators in $B(H)$
        \item $\pi$ is nondegenerate and thus unit-preserving ($\pi(1) = 1$)
    \end{itemize}
\end{definition}

\begin{definition}
    Let $\pi$ be a representation of a C*-algebra $A$ on a Hilbert space $H$. An
    element $\xi$ is called a \textbf{cyclic vector} if the set of vectors
    \begin{equation}
        \{ \pi(x) \cdot \xi : x \in A \}
    \end{equation}
    is norm dense in $H$. Specifically, the range of the representation $\pi(x) \xi$
    is dense in $H$ with respect to the induced norm. When this is the case, $\pi$ is called a \textbf{cyclic representation}.
\end{definition}


The ultimate goal of the GNS construction is to take an abstract state
and realize it as a representation on a Hilbert space. Lets do this step by step.
\begin{enumerate}
    \item We take the algebra $A$ and define a pre-Hilbert space on it where the vectors
    in the pre-Hilbert space are equivalence classes of elements of $A$ with inner-product
    $\langle a, b \rangle_\phi = \phi(b^*a)$ for a state $\phi$ on $A$.
    \item Now the algebra $A$ can act on this pre-Hilbert space by operators. For
    each $a \in A$ we define the operator (representation) $\pi_\varphi (a)$ as acting
    on the Hilbert space $H$ (the completion of the pre-Hilbert space). This representation
    is explicitly realized as $\pi_\varphi(a) \cdot x = a \cdot x$ for some $x$ in
    the Hilbert space $H$.
    \item The construction of $H$ gives us a cyclic vector $\xi \in H$ generated by
    the identity of $A$. This vector is crucial because all of $H$ is generated
    by the action of $A$ onto this vector. Particularly, all vectors in $H$ can
    be written $\pi_\phi(a) \cdot \xi$ for some $a \in A$.
    \item The state $\phi$ corresponds to the expectation of the representations
    $\pi_\phi(a)$ acting on $\xi$. Specifically, $\phi(a) = \langle \pi_\phi(a) \cdot \xi, \xi \rangle$.
\end{enumerate}
Thus our state $\phi$, which started life as a linear functional, can now be realized
as a real-valued expectation on $H$. Now we have all the ingredients we need to
explicitly define the GNS construction.

\begin{theorem}[Gelfand-Naimark-Segal (GNS) Construction]
    Given a state $\phi$ on a C*-algebra $A$, there is a *-representation $\pi$ of $A$ acting
    on a Hilbert space $H$ with a cyclic vector $\xi$ such that
    \begin{equation}
        \phi(a) = \langle \pi(a)\xi , \xi \rangle
    \end{equation}
    for every $a \in A$.
\end{theorem}

\begin{theorem}
    A state $\phi$ on a C*-algebra $A$ is faithful if and only if its GNS representation
    $\pi_\phi: A \to B(H_\phi)$ is injective.
\end{theorem}


\subsection{Examples}
\begin{example}
Lets step through a GNS construction for \( C([0,1]) \) with a simple state. One
can easily verify that $C([0,1])$ is a C*-algebra under pointwise addition, pointwise
multiplication, and the usual suprememum norm. The involution is typically complex
conjugation for continuous functions but because we are only considering real-valued
functions here, we just take the identity involution $f^*=f$. First, lets define a simple state
on $C([0,1])$. Lets consider the state $\varphi$ which is just a point evaluation,
say $x_0 \in [0,1]$, specifically, $\varphi(f) = f(x_0)$. So we see every $f \in C([0,1])$
is a linear functional, we just need to ensure positive definiteness. So we check
$\varphi(f^*f) = f(x_0)^2 \ge 0$. Thus $\varphi$ as we defined it, is a valid state
on $C([0,1])$. Now we need to constuct the appropriate pre-Hilbert space $H_\varphi$.
The elements of $H_\varphi$ will be equivalence classes of functions on $C([0,1])$
with the inner-product given by pointwise multiplication of point evaluations
\begin{equation}
    \langle f,g \rangle_\varphi = \varphi(g^*f) = g(x_0)f(x_0)
\end{equation}
We take the topological closure of $H_\varphi$ to turn it into a proper Hilbert space.
Next, we define a the representation $\pi_\varphi$ of the algebra $C([0,1])$ on
the Hilbert space $H_\varphi$. So we define the operator $\pi_\varphi (f)$ as
\begin{align}
    \pi_\varphi: C([0,1]) &\to B(\mathbb{C})\\
    \pi_\varphi (f)\xi_\varphi &\mapsto f \xi_\varphi
\end{align}
for the cyclic vector $\xi_\varphi$. In this example, $\xi_\varphi$ is any function
that is nonzero when evaluated at $x_0$. For simplicity we can even choose $\xi_\varphi = 1$.
So when we take $f \in C([0,1])$ and apply the cyclic vector $\xi_\varphi$, we get
\begin{equation}
    \pi_\varphi (f) \xi_\varphi = f(x_0) \xi_\varphi
\end{equation}
So the algebra $C([0,1])$ acts on the Hilbert space by multiplication, specifically,
by multiplying the cyclic vector by the function evaluated at $x_0$. Here, our cyclic
vector $\xi_\varphi$ generates the entire Hilbert space under the action of the algebra.
In other words, every vector (function) in $C([0,1])$ can be generated by multiplying
an element $f \in C([0,1])$ by the cyclic vector $\xi_\varphi = 1$. More concretely, for
any function $f \in C([0,1])$, the action on $\xi_\varphi$ is just
\begin{equation}
    \pi_\varphi(f)\xi_\varphi = f(x_0)
\end{equation}
This is somewhat trivial in this example but of course the choice of cyclic vector
can be very difficult in general. Finally, we note that the state  $\varphi$ corresponds
 to the expectation value of the representation acting on the cyclic vector.
Specifically, we have
\begin{equation}
    \varphi(f)\xi_\varphi = \langle \pi_\varphi (f) \xi_\varphi, \xi_\varphi \rangle
\end{equation}
In our case, this boils down to
\begin{equation}
    \varphi(f) = \langle f(x_0), 1 \rangle = f(x_0)
\end{equation}
This is just our function evaluated at $x_0$ which is exactly what our state does.
Despite this being a ``simple'' example, we still had to do a lot of work. When
the choice of pre-Hilbert space and cyclic vector are apparent, the GNS construction
of any C*-algebra will look very similar to this example.
\end{example}


\begin{example}
The GNS construction for \( M_2(\mathbb{C}) \) is still simple but because we are
dealing with matrices, it may give a better intuition of what an arbitray GNS construction
would look like for linear operators. It is easy to see that $M_2(\mathbb{C})$ is a
C*-algebra under the usual matrix addition and multiplication and with involution
given by the conjugate transpose. As for states on $M_2(\mathbb{C})$, there are
a few ``natural'' choices. We will use the trace state
\begin{equation}
    \varphi(A) = \frac{1}{2} | \operatorname{Tr}(A) |
\end{equation}
Note that the trace state is faithful. Using the trace state $\varphi$, we can
build the pre-Hilbert space by taking the inner product
\begin{equation}
    \langle A,B \rangle = \varphi(A^*B) = \frac{1}{2} | \operatorname{Tr}(A^*B) |
\end{equation}
and finding the quotient of $M_2(\mathbb{C})$ by the kernel of the inner-product.
Note that because our action is faithful, $\varphi$ has a trivial null space, which
simplifies the calculations a lot. It is a well known fact that $M_2(\mathbb{C})/\{0\} \cong M_2(\mathbb{C})$.
Thus the pre-Hilbert space is just $M_2(\mathbb{C})$ with the inner product given
by the trace state, which is in fact a Hilbert space. This makes things simple because
we don't need to compute any difficult closures (this can get very cumbersome in general),
essentially saving us a step. Next, we need to define our representation. In this case
it is convenient to take the ``trivial'' representation (the reader may start to see a pattern)
$\pi_\varphi: M_2(\mathbb{C}) \to B(M_2(\mathbb{C}))$
defined by $\pi_\varphi(A)(B) = AB$ for all $A,B \in M_2(\mathbb{C})$
(left multiplication by matrices). This is trivial in the fact that it isn't providing
any additional structure. This representation simply states the fact that matrices act on the Hilbert space
$M_2(\mathbb{C})$ by left multiplication, which is exactly the multiplication operation
we expect. If the reader picked up on the pattern, it will be no surprise that for the
cyclic vector, we just take the identity matrix $\xi = I_2$. So in a concrete sense,
the state representation is
\begin{equation}
    \pi_\varphi (A) \xi = A \xi
\end{equation}
which gives us a natural extension to
\begin{equation}
    \varphi(A) = \langle \pi_\varphi(A )\xi, \xi \rangle = \langle A\xi , \xi \rangle
\end{equation}
which is exactly what we would expect. This example was potentially easier for some
and more difficult for others. But between the two basic examples, hopefully the 
importance of the GNS construction is illustrated. In a nutshell, we can think of 
the GNS construction as a way to express a state on a C*-algebra as an inner
product in a Hilbert space using a representation of the algebra, and a cyclic
vector from the Hilbert space.


\end{example}