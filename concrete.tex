\section{Concrete Examples of States}
\subsection{States on \( C(X) \)}
Recall that for compact space $X$, we denote the C*-algebra of continuous functions
on $X$ by $C(X)$. Fix some point $x_0 \in X$. We define the \textbf{Dirac state} as
$\phi(f) = f(x_0)$. One can easily see that the Dirac state is pure and faithful.
Now, consider the C*-algebra $C^*(G)$ which is the algebra generated by the unitary representations
of a compact group $G$ (we will talk about representations some more in the next section).
Then we define the \textbf{Haar state} as
\begin{equation*}
    \phi(f) = \int_G f(g)d\mu (g)
\end{equation*}
for the Haar measure $\mu$. This state is faithful if $G$ is simple, and tracial if
$G$ is abelian. There are of course many other states on $C(X)$ (and $C^*(X)$) but
these are of particular interest in quantam mechanics and noncommutative geometry.
States on commutative algebras have a very familiar analogue. In fact, for $C(X)$,
states are exactly probability measures. To see this, we define the state
\begin{equation*}
    \phi(f) = \int_X f(x) d \mu(x)   
\end{equation*}
with $\mu(X) = 1$. So under the hood, probability measures on commutative algebras
are states. Now what about the noncommutative case? Consider a, not necessarily commutative,
C*-algebra $A$. If $A$ is noncommutative, then unfortunatley we can't make a generalization
to traditional probability measures. If $A$ is commutative, and thus isomporhic to
$C(X)$ (by Gelfand duality), then the Riesz Representation Theorem guarantees that $\phi$ corresponds
to a probability measure $\mu$ on $X$.


\subsection{States on \( M_2(\mathbb{C}) \)}
In quantam mechanics, a density matrix is a postivie semidefinite operator $\rho$
with operator norm 1. The state corresponding to a given density matrix is
\begin{equation*}
    \phi(A) = \operatorname{Tr}(\rho A)
\end{equation*}
Recall that is $\rho$ is a rank-1 projection, then we can recover the pure state exactly
by taking the inner-product $\phi(A) = \langle \psi, A\psi \rangle$ for a unit vector
$\psi \in H$. If $\rho$ is not rank-1, then it is a mixed state. These mixed states
have a special correspondence to the probability of quantam states which we will
discuss later.

\subsection{States on $B(H)$}
For noncommutative C*-algebras, the theory changes but there is still a relationship
to probability. Consider the C*-algebra of bounded linear operators on a Hilbert
space $B(H)$. Recall that  the involution on $B(H)$ is the adjoint operator. For a
quantum state $\phi$ on $B(H)$, we can define a probability distribution for each
observable. In this context, observables are self-adjoint elements of $B(H)$ ($A=A^*$).
Suppose we have an observable $A$ with the given spectral decomposition
\begin{equation*}
    A = \sum_i \lambda_i P_i
\end{equation*}
where the $P_i$ are projections onto the eigenspaces of $A$. Then we can find the 
probability of a given $\lambda_i$ by $\mathbb{P}(A=\lambda_i) = \phi(P_i)$. So the
quantum state doesn't assign a probability distribution to the whole space, but rather
assigns a probability distribution to each observable in $B(H)$. If we have two
observables $A$ and $B$ that do not commute, then there does not exist a single 
probability distribution that describes both simultaneously.

\par 

In summary, states on commutative C*-algebras correspond to probability measures
either directly or by the Riesz Representation Theorem. States on noncommutative
C*-algebras cannot assign a global probability measure, but rather each observable
is assigned a probability measure. So the commutative case lets us make a global
generalization while the noncommutative case only allows local consistency. We have
only just scratched the surface of states here but hopefully the reader can start
to understand the importance they have in quantum mechanics and noncommutative geometry.

